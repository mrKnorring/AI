\documentclass[12pt, a4paper]{article}
\usepackage[english]{babel}
\usepackage[utf8]{inputenc}
\usepackage{graphicx}
\usepackage{hyperref}
\usepackage{listings}

\title{Artificiell intelligens\\First lab}
\author{Simon von Knorring\\Jesper Svennebring}

%\begin{figure}[ht]
%\centering
%\includegraphics[width=0.4\linewidth]{knorring.jpg}
%\end{figure}
\begin{document}
\maketitle

\newpage
\tableofcontents

\newpage
\section{A theoretical overview of the A* algorithm}
% A theoretical overview of the A* algorithm, including an explanation of optimality conditions.
The A* algorithm is a tree or node search, that looks for the cheapest way to get from one place to another.\\\\

Every node has a specific H-value, which is the distance between a start node and the given goal node.\\\\

Every node also has a G-value, which is calculated by adding the total cost of the cheapest way to get to that node from the start node.\\\\

Finally every node has an F-value that is the sum of the G and H value of the node. This value represents how attractive it is to move to this node.\\\\

We also have two lists called open and closed list, where open list contains all nodes that we have calculated the H, G and F values for but that we haven’t analyzed the neighbors of, closed list is where we put the nodes that we have used to calculate the neighbors H, G, F value in.\\\\

Lastly every node keeps track of witch other neighboring node that gives the lowest G value when used to traveling to it in other words what node is its “parent”.\\\\ 

We start the algorithm of with calculating all nearby nodes H, G, F, we then check if they are in the open list, if not we put them in and give them all the current node as a “parent”, else we check if their current F value is lower than the F value of the node we are standing on, if it is lower we switch them. All nodes that were not in the open list we put in the open list. After that we look in the open list for witch node has the lowest F value, and then starts over with that value. 


\section{Our A*}
%A discussion of the A* search algorithm you implemented, including a discussion of your heuristic, and whether you used a tree or graph search. Explain why you made the choices you made
We have made a function that uses the theory of A * algorithm to update our open and closed list. When it terminates, it return the closed list.\\
The first thing we do is to search to search in our open list after the node with the best F-value. Meaning the lowes value.\\
Then we search in the open list after the node with the best F-value and remove it from the open list to the closed list. And we store it as the last item in the list.\\
Then we do a check at that new node if it is the goal node.If it is we return the closedlist. If not we add the neighbors of that node to the open list, assuming that they are not already in the list.\\
Lastly recursive this procedure, til we found the goal node.\\\\

As you can see, we uses a graph search. Because we believe that it is the most effective one, depending on our situation. Even though it store more memory, it is better in the longer run.

\begin{lstlisting}[language=R]
AStar=function(roads,openlist,closedlist,goalNode) {
  current = list()
  bestF = 10000
  for(node in openlist){
    if(node$F < bestF){
      bestF = node$F
    } 
  }
  for(i in 1:length(openlist)){
    node = openlist[[i]]
      if(node$F == bestF){
        closedlist[[length(closedlist)+1]] = node
        current = node
        openlist[[i]] = NULL
        break
      } 
    }
  if(closedlist[[length(closedlist)]]$x == goalNode$x && 
    closedlist[[length(closedlist)]]$y == goalNode$y){
    return(closedlist) 
  }
  
  if(current$x != 1){
    newX = current$x -1
    leftNodeCoor = list(x = newX, y = current$y)
    leftNode = getNewNode(leftNodeCoor, current, roads, goalNode)
    openlist = openlistAdd(openlist, leftNode)
  } 
  
  if(current$y != nrow(roads$hroads)){
    newY = current$y + 1
    topNodeCoor = list(x = current$x, y = newY)
    topNode = getNewNode(topNodeCoor, current, roads, goalNode)
    openlist = openlistAdd(openlist, topNode)
  } 
  
  if(current$x != ncol(roads$vroads)){
    newX = current$x + 1
    rightNodeCoor = list(x = newX, y = current$y)
    rightNode = getNewNode(rightNodeCoor, current, roads, goalNode)
    openlist = openlistAdd(openlist, rightNode)
  } 
  if(current$y != 1){
    newY = current$y -1
    bottomNodeCoor = list(x = current$x, y = newY)
    bottomNode = getNewNode(bottomNodeCoor, current, roads, goalNode)
    openlist = openlistAdd(openlist, bottomNode)
  }  
  AStar(roads,openlist,closedlist,goalNode)
}

\end{lstlisting}

% Graphical search

% Extras

\section{Strategies we made}
% A discussion of not A* strategies you made us of to improve your performance. Explain why you used these strategies.
% Kollar vilken grön nod som ligger närmas
We have made two assumptions that is worth mentioning. First, we assume that the roads get new values op on doing an action, rather than that they change every time the turn counter increase by one, we are how ever not completely sure it works this way. The other assumption is that the world will get more and more corrupt the longer we take to finish, in other words, we assume that the program will terminate long before the roads in the world starts to stabilize  .\\

At the moment we are doing a rather simple check to see what where the nearest package is by checking all green nodes H value and then going towards the one that has the lowest H value. This way of finding the optimal rout is far from perfect but we still think that comparing H values is just about as good as checking witch green node has the best F value, this due to our assumption that the state of the world will get worse over time, resulting in that the shorter rout we take, the faster our later green packages should take to get on average.\\

We think that the best way to optimize the order we pick up and deliver packages would be to make a check of how packages pickup and delivery points lies compared to others, and then try to find the cheapest way to pick up all packages by simply checking all possible routs to see witch one would be cheapest. This could be done only once if one wants to be time effective or between every move if one wants to account for changes on the board. Worth noting that accounting for changes will probably give a better average, even if it would be a lot less time effective. The optimal route would probably be calculated by taking the state of the board in to account, but also modifying it a bit, by prioritizing shorter routs above cheaper routs in the start, and slowly switching focus to cheaper over short the closer to the end of the run one gets.\\

A change to the normal A* algorithm we made was to increase G by 2 per node forward we calculate. Witch gave us CurrentNodeG = ParentG + CostToMoveThere + 2, normally we would only have had +1 to calculate for the cost of doing an action. Instead we increase it by +2 to give prioritization to give a bit extra prioritization to the shortest rout, the optimal value would probably be some where between +1 and +2, and having it shift closer and closer to +1 the closer we are to the end of the entire run.


\end{document}